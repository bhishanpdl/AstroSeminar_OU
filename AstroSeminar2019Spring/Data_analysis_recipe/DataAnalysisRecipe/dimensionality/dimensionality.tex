% This file is part of the Data Analysis Recipes project.
% Copyright 2013 the author(s).

% to-do
% -----
% - outline
% - write
% - put on arXiv

\documentclass[12pt,twoside]{article}
\usepackage{amssymb}
\usepackage{amsmath}
\usepackage{mathrsfs}
\IfFileExists{./hogg_endnotes.sty}{\usepackage{./hogg_endnotes}}{}
\IfFileExists{../hogg_endnotes.sty}{\usepackage{../hogg_endnotes}}{}
\usepackage{natbib}
\usepackage{float}
\usepackage{graphicx}
\usepackage{multirow}

%%Figure caption
\makeatletter
\newsavebox{\tempbox}
\newcommand{\@makefigcaption}[2]{%
\vspace{10pt}{#1.--- #2\par}}%
\renewcommand{\figure}{\let\@makecaption\@makefigcaption\@float{figure}}
\makeatother

\newcommand{\exampleplot}[1]{%
\begin{center}%
\includegraphics[width=0.5\textwidth]{#1}%
\end{center}%
}
\newcommand{\exampleplottwo}[2]{%
\begin{center}%
\includegraphics[width=0.5\textwidth]{#1}%
\includegraphics[width=0.5\textwidth]{#2}%
\end{center}%
}

\setlength{\emergencystretch}{2em}%No overflow

\newcommand{\notenglish}[1]{\textsl{#1}}
\newcommand{\aposteriori}{\notenglish{a~posteriori}}
\newcommand{\apriori}{\notenglish{a~priori}}
\newcommand{\adhoc}{\notenglish{ad~hoc}}
\newcommand{\etal}{\notenglish{et al.}}
\newcommand{\eg}{\notenglish{e.g.}}
\newcommand{\project}[1]{\textsl{#1}}

\newcommand{\documentname}{document}
\newcommand{\documentnames}{\documentname s}
\newcommand{\sectionname}{Section}
\newcommand{\equationname}{equation}
\newcommand{\Equationname}{Equation}
\newcommand{\equationnames}{\equationname s}
\newcommand{\figurenames}{\figurename s}
\newcommand{\problemname}{Exercise}
\newcommand{\problemnames}{\problemname s}
\newcommand{\solutionname}{Solution}
\newcommand{\notename}{note}
\renewcommand{\and}{{\footnotesize{and}}}

\newcommand{\note}[1]{\endnote{#1}}
\def\enotesize{\normalsize}
\renewcommand{\thefootnote}{\fnsymbol{footnote}} % the ONE footnote needs this

\newcounter{problem}
\newenvironment{problem}{\paragraph{\problemname~\theproblem:}\refstepcounter{problem}}{}
\newcommand{\affil}[1]{{\footnotesize\textsl{#1}}}

% matrix stuff
\newcommand{\mmatrix}[1]{\boldsymbol{#1}}
\newcommand{\inverse}[1]{{#1}^{-1}}
\newcommand{\transpose}[1]{{#1}^{\scriptscriptstyle \top}}
\newcommand{\mA}{\mmatrix{A}}
\newcommand{\mAT}{\transpose{\mA}}
\newcommand{\mC}{\mmatrix{C}}
\newcommand{\mCinv}{\inverse{\mC}}
\newcommand{\me}{\mmatrix{e}}
\newcommand{\mQ}{\mmatrix{Q}}
\newcommand{\mS}{\mmatrix{S}}
\newcommand{\mX}{\mmatrix{X}}
\newcommand{\mY}{\mmatrix{Y}}
\newcommand{\mYT}{\transpose{\mY}}
\newcommand{\mZ}{\mmatrix{Z}}
\newcommand{\vhat}{\mmatrix{\hat{v}}}

% parameter vectors
\newcommand{\parametervector}[1]{\mmatrix{#1}}
\newcommand{\pvtheta}{\parametervector{\theta}}

% set stuff
\newcommand{\setofall}[3]{\{{#1}\}_{{#2}}^{{#3}}}
\newcommand{\allq}{\setofall{q_i}{i=1}{N}}
\newcommand{\allx}{\setofall{x_i}{i=1}{N}}
\newcommand{\ally}{\setofall{y_i}{i=1}{N}}
\newcommand{\allxy}{\setofall{x_i,y_i}{i=1}{N}}
\newcommand{\allsigmay}{\setofall{\sigma_{yi}^2}{i=1}{N}}
\newcommand{\allS}{\setofall{\mS_i}{i=1}{N}}
\newcommand{\union}{\cup}
\newcommand{\intersect}{\cap}

% other random multiply used math symbols
\DeclareMathOperator*{\argmax}{arg\,max}
\DeclareMathOperator*{\argmin}{arg\,min}
\renewcommand{\d}{\mathrm{d}}
\newcommand{\dd}{\d}
\newcommand{\mean}[1]{\left<{#1}\right>}
\newcommand{\like}{\mathscr{L}}
\newcommand{\given}{\,|\,}
\newcommand{\atfixed}{\,;}
\newcommand{\best}{\mathrm{best}}
\newcommand{\normal}{\mathscr{N}}
\newcommand{\Pbad}{P_{\mathrm{b}}}
\newcommand{\Ybad}{Y_{\mathrm{b}}}
\newcommand{\Vbad}{V_{\mathrm{b}}}
\newcommand{\pfg}{p_{\mathrm{fg}}}
\newcommand{\pbg}{p_{\mathrm{bg}}}
\newcommand{\bperp}{b_{\perp}}
\newcommand{\meanZ}{\mean{\mZ}}


% header stuff
\renewcommand{\MakeUppercase}[1]{#1}
\pagestyle{myheadings}
\renewcommand{\sectionmark}[1]{\markright{\thesection.~#1}}
\markboth{Dimensionality reduction}{}

\newcommand{\data}{D}
\newcommand{\pars}{\theta}
\newcommand{\cv}{\mathrm{CV}}
\newcommand{\pcv}{p_{\cv}}

\begin{document}
\thispagestyle{plain}\raggedbottom
\section*{Data analysis recipes:\ \\
  Data-driven models and reduction of \\ dimensionality}

\footnotetext{The \notename s begin on page~\pageref{note:first},
  including the license\note{\label{note:first} Copyright 2013 by the
    authors.  You may copy and distribute this document provided that
    you make no changes to it whatsoever.}  and the
  acknowledgements\note{It is a pleasure to thank
      Sam Roweis (deceased), and
      Vivi Tsalmantza (Heidelberg)
    for valuable comments and discussions.  This research was
    partially supported by NASA (ADP grant NNX08AJ48G), NSF (grant
    AST-0908357), and a Research Fellowship of the Alexander von
    Humboldt Foundation.}}

\noindent
David~W.~Hogg\\
\affil{Center~for~Cosmology~and~Particle~Physics, Department~of~Physics, New~York~University}\\
\affil{Max-Planck-Institut f\"ur Astronomie, Heidelberg}

\begin{abstract}
There are many situations in which investigators have large amounts of
data from some phenomenon, but no good model; in these cases, a model
can be built from the data themselves.  Because these kinds of models
are designed to produce compact approximate representations of the
data, they are forms of dimensionality reduction.  The standard and
simplest method is that of principal components analysis (PCA), which
is effective with data sets in which the structure of interest is
detected at high signal-to-noise in all data points, and all data
points are completely measured.  I describe PCA and its great
limitations, and suggest a number of methods that are far better and
not far more complex.  In particular I focus on methods that permit
using data with large noise, heteroskedasticity, and missing values;
these methods are probabilistic and generative.
\end{abstract}

\clearpage
\markright{Notes}\theendnotes

\clearpage
\begin{thebibliography}{}\markright{References}
\bibitem[Bovy, Hogg, \& Roweis(2009)]{bovy}
  Bovy,~J., Hogg,~D.~W., \& Roweis, S.~T., 2009,
  Extreme deconvolution: inferring complete distribution functions from noisy, heterogeneous, and incomplete observations, 
  arXiv:0905.2979 [stat.ME]
\end{thebibliography}

\end{document}
