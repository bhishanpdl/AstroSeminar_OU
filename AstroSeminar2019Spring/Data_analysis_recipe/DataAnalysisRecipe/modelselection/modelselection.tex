% This file is part of the Data Analysis Recipes project.
% Copyright 2011, 2012, 2013 the author(s).

% to-do
% -----
% - make figures demonstrating techniques
% - the figures should demonstrate all three of AIC/BIC, cross-validation, and marginalized likelihood
% - the continuously variable complexity example takes up too much space
% - the continuously variable complexity example needs to switch to a cubic spline model
% - argument against the evidence needs to be sharpened up.  See email to Ivezic and Vanderplas in 2013-05.

\documentclass[12pt,twoside]{article}
\usepackage{amssymb}
\usepackage{amsmath}
\usepackage{mathrsfs}
\IfFileExists{./hogg_endnotes.sty}{\usepackage{./hogg_endnotes}}{}
\IfFileExists{../hogg_endnotes.sty}{\usepackage{../hogg_endnotes}}{}
\usepackage{natbib}
\usepackage{float}
\usepackage{graphicx}
\usepackage{multirow}

%%Figure caption
\makeatletter
\newsavebox{\tempbox}
\newcommand{\@makefigcaption}[2]{%
\vspace{10pt}{#1.--- #2\par}}%
\renewcommand{\figure}{\let\@makecaption\@makefigcaption\@float{figure}}
\makeatother

\newcommand{\exampleplot}[1]{%
\begin{center}%
\includegraphics[width=0.5\textwidth]{#1}%
\end{center}%
}
\newcommand{\exampleplottwo}[2]{%
\begin{center}%
\includegraphics[width=0.5\textwidth]{#1}%
\includegraphics[width=0.5\textwidth]{#2}%
\end{center}%
}

\setlength{\emergencystretch}{2em}%No overflow

\newcommand{\notenglish}[1]{\textsl{#1}}
\newcommand{\aposteriori}{\notenglish{a~posteriori}}
\newcommand{\apriori}{\notenglish{a~priori}}
\newcommand{\adhoc}{\notenglish{ad~hoc}}
\newcommand{\etal}{\notenglish{et al.}}
\newcommand{\eg}{\notenglish{e.g.}}
\newcommand{\project}[1]{\textsl{#1}}

\newcommand{\documentname}{document}
\newcommand{\documentnames}{\documentname s}
\newcommand{\sectionname}{Section}
\newcommand{\equationname}{equation}
\newcommand{\Equationname}{Equation}
\newcommand{\equationnames}{\equationname s}
\newcommand{\figurenames}{\figurename s}
\newcommand{\problemname}{Exercise}
\newcommand{\problemnames}{\problemname s}
\newcommand{\solutionname}{Solution}
\newcommand{\notename}{note}
\renewcommand{\and}{{\footnotesize{and}}}

\newcommand{\note}[1]{\endnote{#1}}
\def\enotesize{\normalsize}
\renewcommand{\thefootnote}{\fnsymbol{footnote}} % the ONE footnote needs this

\newcounter{problem}
\newenvironment{problem}{\paragraph{\problemname~\theproblem:}\refstepcounter{problem}}{}
\newcommand{\affil}[1]{{\footnotesize\textsl{#1}}}

% matrix stuff
\newcommand{\mmatrix}[1]{\boldsymbol{#1}}
\newcommand{\inverse}[1]{{#1}^{-1}}
\newcommand{\transpose}[1]{{#1}^{\scriptscriptstyle \top}}
\newcommand{\mA}{\mmatrix{A}}
\newcommand{\mAT}{\transpose{\mA}}
\newcommand{\mC}{\mmatrix{C}}
\newcommand{\mCinv}{\inverse{\mC}}
\newcommand{\me}{\mmatrix{e}}
\newcommand{\mQ}{\mmatrix{Q}}
\newcommand{\mS}{\mmatrix{S}}
\newcommand{\mX}{\mmatrix{X}}
\newcommand{\mY}{\mmatrix{Y}}
\newcommand{\mYT}{\transpose{\mY}}
\newcommand{\mZ}{\mmatrix{Z}}
\newcommand{\vhat}{\mmatrix{\hat{v}}}

% parameter vectors
\newcommand{\parametervector}[1]{\mmatrix{#1}}
\newcommand{\pvtheta}{\parametervector{\theta}}

% set stuff
\newcommand{\setofall}[3]{\{{#1}\}_{{#2}}^{{#3}}}
\newcommand{\allq}{\setofall{q_i}{i=1}{N}}
\newcommand{\allx}{\setofall{x_i}{i=1}{N}}
\newcommand{\ally}{\setofall{y_i}{i=1}{N}}
\newcommand{\allxy}{\setofall{x_i,y_i}{i=1}{N}}
\newcommand{\allsigmay}{\setofall{\sigma_{yi}^2}{i=1}{N}}
\newcommand{\allS}{\setofall{\mS_i}{i=1}{N}}
\newcommand{\union}{\cup}
\newcommand{\intersect}{\cap}

% other random multiply used math symbols
\DeclareMathOperator*{\argmax}{arg\,max}
\DeclareMathOperator*{\argmin}{arg\,min}
\renewcommand{\d}{\mathrm{d}}
\newcommand{\dd}{\d}
\newcommand{\mean}[1]{\left<{#1}\right>}
\newcommand{\like}{\mathscr{L}}
\newcommand{\given}{\,|\,}
\newcommand{\atfixed}{\,;}
\newcommand{\best}{\mathrm{best}}
\newcommand{\normal}{\mathscr{N}}
\newcommand{\Pbad}{P_{\mathrm{b}}}
\newcommand{\Ybad}{Y_{\mathrm{b}}}
\newcommand{\Vbad}{V_{\mathrm{b}}}
\newcommand{\pfg}{p_{\mathrm{fg}}}
\newcommand{\pbg}{p_{\mathrm{bg}}}
\newcommand{\bperp}{b_{\perp}}
\newcommand{\meanZ}{\mean{\mZ}}


% header stuff
\renewcommand{\MakeUppercase}[1]{#1}
\pagestyle{myheadings}
\renewcommand{\sectionmark}[1]{\markright{\thesection.~#1}}
\markboth{Comparing models of different complexity}{}

\newcommand{\data}{D}
\newcommand{\pars}{\theta}
\newcommand{\cv}{\mathrm{CV}}
\newcommand{\pcv}{p_{\cv}}

\begin{document}
\thispagestyle{plain}\raggedbottom
\section*{Data analysis recipes:\ \\
  Comparing models of different complexity\footnotemark}

\footnotetext{The \notename s begin on page~\pageref{note:first},
  including the license\note{\label{note:first} Copyright 2011, 2012, 2013 by the
    authors.  You may copy and distribute this document provided that
    you make no changes to it whatsoever.}  and the
  acknowledgements\note{It is a pleasure to thank
      Jo Bovy (IAS),
      Dustin Lang (CMU),
      Dan Foreman-Mackey (NYU),
      Iain Murray (Edinburgh),
      Hans-Walter Rix (MPIA), and
      Sam Roweis (deceased)
    for valuable comments and discussions.  This research was
    partially supported by NASA (ADP grant NNX08AJ48G), NSF (grant
    AST-0908357), and a Research Fellowship of the Alexander von
    Humboldt Foundation.  This research made use of the Python
    programming language and the open-source Python packages scipy,
    numpy, and matplotlib.}.}

\noindent
David~W.~Hogg\\
\affil{Center~for~Cosmology~and~Particle~Physics, Department~of~Physics, New~York~University}\\
\affil{Max-Planck-Institut f\"ur Astronomie, Heidelberg}
%% \\[1ex]
%% Jo~Bovy\\
%% \affil{Institute for Advanced Study, Princeton}
%% \\[1ex]
%% Dustin~Lang\\
%% \affil{Princeton University Observatory}\\
%% \affil{Department of Physics, Carnegie Mellon University}

\begin{abstract}
  Frequently the data analyst is faced with the comparison of models
  of very different levels of complexity or very different amounts of
  freedom to fit or explain the data.  We begin by discussing the
  assessment of model complexity, in part to criticize the idea that
  it is directly related to the number of parameters in the model: It
  is not, except in rare circumstances.  Model selection---choosing
  among models of differing complexity---can often be avoided; we
  advocate avoiding it.  If the decision cannot be avoided, we discuss
  options, including frequentist information criteria, Bayesian
  ``evidence'', and probabilistic decision theory.  I recommend the
  data-driven technique of leave-one-out cross-validation, which
  locates the model that best predicts left-out ``validation'' data
  after being fit to (or learning from) left-in ``training'' data.
  Ultimately, decisions such as model-selection decisions must involve
  the investigator's utility; all methods that ignore utility (as all the above-mentioned methods do) are
  bound to lead to unwanted results at least sometimes.
\end{abstract}

A scientist makes model complexity decisions---implicitly or
explicitly---every day.  For example, on a trivial scale, any decision
about binning a histogram (small bins or big bins) or fitting a curve
(linear or quadratic) or combining data (average all the data taken on
the same day or all the data taken in the same year) is a decision
about model complexity; in each case the decision is about how much
freedom to give to some free function.

At a grander scale, many important scientific discoveries are model
complexity decisions.  For example, the decision that data on a star's
radial velocity as a function of time is better described by a model
that includes perturbations from an orbiting exoplanet---better than a
model that \emph{doesn't} include any exoplanet---is simultaneously a
model selection decision and an exoplanet \emph{discovery}.  That
discovery is a decision to increase model complexity, but also many
scientific discoveries are decisions to decrease model complexity;
consider for example the heliocentric model of the Solar System.  It
was adopted in part because it led to an enormous decrease in the
complexity of the quantitative Solar System model, reducing greatly
the number of free parameters.

Of course a model with more freedom will, in general, lead to a better
description of the data in the trivial sense that it can reproduce the
observed features more closely.  But a model with more freedom will
also, in general, be less predictive, because finite uncertainties in
the parameters will propagate into greater uncertainty in prediction
as the number of imprecisely determined (independent) parameters
increases.  The challenge of model selection is to balance these
considerations; a scientifically good model must both describe the
existing data well and also predict new or left-out data well.\note{In
  this sense, Ockham's Razor, na\"ively construed, is wrong, or
  perhaps useless: It is not the case that the simplest model that
  explains the data is the best.  First of all, the scientist is
  (almost) never presented with two models that are \emph{equally}
  good at explaining the data.  Second of all, even if the two models
  \emph{are} equal at explaining existing data, are they equal at
  predicting new data?  Some Bayesians have taken Ockham's Razor very
  seriously, claiming that Bayesian inference and Bayesian model
  selection justifies Ockham's Razor, but this view relies on a
  careful interpretation of Ockham that doesn't necessarily do justice
  to the original.}

One key idea in what follows is that probability alone---the rules
underlying the frequentist and Bayesian reasoning we have been
tracking in these \documentnames ---cannot tell you how to strike the
balance between model freedom and goodness-of-fit.  How you decide
between models involves probability, to be sure, but it also involves
your \emph{utility}.  For example, in the exoplanet example mentioned
above, whether or not you decide that there is an exoplanet in the
data depends in part on how much the exoplanet improves the fit.  But
it also depends in part on how much it hurts you if you claim a
discovery and later turn out to be wrong.  Or how much it hurts you to
\emph{miss} making a discovery that you might have made.  I won't say,
in this document, all that much about how you compute your utility;
that's for another \documentname\note{cite decision \documentname},
but we will see where utility comes in to the calculation, and the
different things that a Bayesian or frequentist should do with it.

Even the most trivial decisions in a data analysis pipeline (for
example, decisions about what data to include or not in a catalog or
table, what data to mark as ``outliers'' or corrupted, and so on) are
themselves usually model-complexity decisions.  They tend to be made
casually and heuristically, on the basis of things like the ``number
of sigma'' some parameter is out of spec, or likelihood ratios (or,
equivalently, $\chi^2$ differences).  These choices are all
legitimate; they will all come up again below.  These heuristics have
been discovered because they give good approximations to utility
optimizations, or because they give good protection from utility
disasters.

\section{Model complexity}

The concept of model complexity is somewhat slippery.  In the case of
a purely linear model, the model complexity is directly related to the
number of free parameters, but in every other case it is less
well-defined.  We will begin with a few examples of models of
differing or variable complexity to get at some of the issues in the
\emph{definition} of model complexity.  After that, we will turn to
methods for choosing among models of different complexity.

The most simple case is that of a linear model---a model in which the
predicted mean of the data is a linear function of the
parameters---and Gaussian noise of known variances.  In this case,
logarithmic likelihood ratios are just $\chi^2$ differences and model
complexity is directly associated with the \emph{number of parameters}
$K$.  The more parameters, the more freedom the model has; it is that
simple.  The only assumptions behind this (beyond linearity and
Gaussianity and known variances) is that the linear functions are
linearly independent in the vector sense\note{Here the over-used word
  ``independent'' means can not be coadded with non-zero coefficients
  to produce zero mean prediction; this is independence in the vector
  sense of spanning a full-rank space, not the probability sense of
  separability.} and that they are non-trivial in the sense that they
have support or act at the location of the data\note{Strictly this
  support requirement is implied by independence already.}.

Most interesting models are \emph{not} linear however, so this simple
relationship between the number of parameters $K$ and the model
complexity should always be treated with suspicion and is not true in
general.  One model that serves as an example to illustrate this
divergence between the complexity and the number of parameters is the
following.\note{We owe this model to Yann LeCun (NYU), via Sam
  Roweis.}  Imagine having $N$ data points $(x_i,y_i)$ in some finite
domain in $x$ and $y$, with observational uncertainties of some kind
we need not specify precisely for the moment.  Now fit to these data
the model
\begin{equation}\label{eq:lecun}
y_i = A\,\sin (k\,x + \phi)
  \quad ,
\end{equation}
where $A$ is an amplitude, $k$ is a wave number (or frequency), and
$\phi$ is a phase.  With suitable---and suitably
\emph{precise}---specification of these parameters, it is possible to
\emph{exactly fit all $N$ data points, no matter what they are}.  The
only assumption is that $x_i \ne x_j$ for $i \ne j$.  Furthermore,
there is not just one location in the $(A,k,\phi)$ space that fits all
the data points, there is an enormous infinity of solutions, no matter
how large $N$.

In the standard situation of linear fitting, if there are $K$
components---and thus $K$ free parameters---it is possible to match
\emph{exactly} at most $K$ data points, and even this is only
guaranteed if all of the components have support on the $x$ range of
the data and they are appropriately independent under the $x$-sampling
of the data.  If we equate model complexity with the power to flexibly
fit arbitrary data, then the three-parameter sinusoidal model of
\equationname~(\ref{eq:lecun}) has as much complexity as a linear
model with an \emph{infinite number of parameters}.

Of course the implied paradox has a simple resolution: In order to fit
the data precisely, the parameters of the model---especially the wave
number $k$---must be specified to enormous \emph{precision} or \emph{bit depth}.  That is,
the $K$-component linear model gets its power to fit by having a large
number of parameters.  The sinusoidal model gets its power to fit by
having a large number of digits of accuracy devoted to the $k$
parameter (and, in some limits, the $A$ parameter).

...Hogg:  Perhaps give a specific illustration of this here.---Hogg

When a linear model with $K=N$ is used to exactly fit all $N$ data
points, the $N$ parameters can be seen as a simple invertible linear
transformation of the data; that is, the parameters are (in some
sense) a lossless encoding of the data.  When the sinusoidal model is
used to exactly fit all $N$ points, the model again is being used to
losslessly encode the data, but now the encoding is nonlinear, and
requires of the parameters enormous precision in their representation.
This enormous precision is required because all of the information in
all $N$ data points---all the bits, if you like---must be encoded into
only three parameters.

Another model that serves to illustrate that model complexity is not
directly related to the number of parameters is the following, in
which there are \emph{far more parameters} than data points, but
nonetheless the model has limited freedom.  The freedom of the model
is controlled by a continuous smoothness parameter, which effectively
makes this a model with continuously variable complexity.

Imagine a set of $N$ data points $(x_i,y_i)$ which deviate from a
$K$-dimensional (or $K$-parameter) linear model by the addition (to
the $y$ values) of noise samples $e_i$ drawn from Gaussian
distributions of (presumed known) variances $\sigma_{yi}^2$.  The
model can be written as
\begin{equation}
y_i = \sum_{j=1}^K
  a_{ij}\,x_j + e_i
  \quad ,
\end{equation}
or, in matrix notation,
\begin{equation}
\mY = \mA\,\mX + \me
  \quad ,
\end{equation}
where $\mY$ is an $N$-dimensional vector of the $y_i$, $\mA$ is an
$N\times K$ matrix of (presumed known) coefficients $a_{ij}$, $\mX$ is
a $K$-dimensional vector of the parameters $x_j$, and $\me$ is an
$N$-dimensional vector of the noise draws or errors $e_i$.  The
standard thing to do is to minimize a $\chi^2$ scalar
\begin{equation}
\chi^2 = \sum_{i=1}^N
  \frac{\left[y_i - a_{ij}\,x_j\right]^2}{\sigma_{yi}^2}
  \quad ,
\end{equation}
where this is scientifically justified when all the aforementioned
assumptions (linear model space includes the truth, noise is Gaussian
with known variances, and so on) are true.  This objective ($\chi^2$)
is truly a scalar in the sense that it has a compact matrix
representation
\begin{equation}
\chi^2 = \transpose{\left[\mY-\mA\,\mX\right]}
  \,\mCinv\,\left[\mY-\mA\,\mX\right]
  \quad .
\end{equation}
In this situation, when $N<K$, the best-fit value for the parameters
(the elements of $\mX$) is given by
\begin{equation}
\mX_\best = \inverse{\left[\mAT\,\mCinv\,\mA\right]}
  \,\left[\mAT\,\mCinv\,\mY\right]
  \quad ,
\end{equation}
which was found by forcing the full derivative of the $\chi^2$ scalar
to vanish.  In this formulation, if the (errors of the) $y_i$ are
independent, the covariance matrix $\mC$ is the diagonal $N\times N$
matrix with the noise variances $\sigma_{yi}^2$ on the diagonal.

Frequently (for engineers, if not astronomers) the condition $N<K$
does not hold, but there are smoothness expectations that provide
enough support to set the parameters nonetheless.  The simplest
implementation of a smoothness prior or regularization is the addition
to $\chi^2$ of a quadratic penalty for derivatives in the parameters
$x_j$.  In this picture, we imagine the parameters $x_j$ are ordered
such that nearby $j$ are nearby in the appropriate sense.  The scalar
objective gets modified in this case to
\begin{equation}
\chi_r^2 = \sum_{i=1}^N
  \frac{\left[y_i - a_{ij}\,x_j\right]^2}{\sigma_{yi}^2}
  + \epsilon\,\sum_{j=1}^{K-1}\left[x_{j+1}-x_j\right]^2
  \quad ,
\end{equation}
where $\epsilon$ is a control parameter that controls the smoothness
of the model, or the stiffness of it, or else the model complexity in
some sense.  Under this objective, the best fit parameter vector is
given by
\begin{equation}
\mX_\best = \inverse{\left[\mAT\,\mCinv\,\mA + \epsilon\,\mQ\right]}
  \,\left[\mAT\,\mCinv\,\mY\right]
  \quad ,
\end{equation}
where the $K\times K$ matrix $\mQ$ is a tri-diagonal matrix that looks
like this
\begin{equation}
\mQ = \left[\begin{array}{cccccc}
    1 &-1 & 0 & 0 & 0 & 0 \\
   -1 & 2 &-1 & 0 & 0 & 0 \\
    0 &-1 & 2 &-1 & 0 & 0 \\
    0 & 0 &-1 & 2 &-1 & 0 \\
    0 & 0 & 0 &-1 & 2 &-1 \\
    0 & 0 & 0 & 0 &-1 & 1 \\
  \end{array}\right]
  \quad ,
\end{equation}
in the case of $K=6$ parameters and generalizes in the obvious way for
different $K$.

\section{What is a model?}

Enough about model complexity or model freedom; it is time to
\emph{select} the best model from a set.  That said, before we can
talk about deciding \emph{among} models, we have to say what a model
\emph{is}.  We have covered this ground in previous
\documentnames\note{cite SOME OTHER DOCUMENT.}, but we will briefly
recap here.  For our purposes, a model is a \emph{prediction about
  data}.  That prediction needs to include, at the very least, a
probability density function (pdf) $p(\data\given\pars, H)$ that
assigns a probability density to any possible data set $\data$ given a
setting of parameters $\pars$ and some fundamental model assumptions
and approximations and prior information $H$.  This picture of a model
is aggressively probabilistic: It produces probabilities over data,
not data.  In addition, a model often (Bayesians would say
\emph{always}) also has a prior pdf over model parameters
$p(\pars\given H)$.

A few notes: The pdf for the data $p(\data\given\pars, H)$ is often
called (dumbly\note{explain dumbness}) the ``likelihood'' for the
parameters, especially when it is evaluated at the value of the
actual, observed data and treated as a function of the parameters
$\pars$.  The data object $\data$ should be thought of as a
high-dimensional vector or list or blob of data, with one element for
every scalar measurement or observation.  The parameters object
$\pars$ is similarly a high-dimensional vector or list or blob of all
the parameters that are permitted to vary.  The model-specification
object $H$ (the contents of which we will never fully specify\note{It
  would be almost impossible to list everything that is absorbed into
  $H$, since in some sense it includes an enumeration of everything
  you are assuming, explicitly and implicitly, in your model.  This
  includes crazy things like that your computer obeys the IEEE
  arithmetic specification, and that the data file you have was not
  generated by a trickster demigod.}) includes all the model
assumptions, but also all the prior information like the noise
properties of the data and so on.  This huge bundle of information is
all part of the model in the sense that different assumptions will
lead to different likelihood functions and therefore different
calculations of probabilities.  As we emphasize elsewhere\note{cite
  PROBABILITY CALCULUS DOCUMENT}, the likelihood has units of
inverse-data, the prior pdf has units of inverse-parameters; these
dimensional facts will be relevant when we do integrals (below).

For what follows, it is useful to remember that, when the noise
contributions are treated as being drawn from Gaussians with known
variances, the logarithm of the likelihood is proportional to a
quadratic $\chi^2$ objective.\note{Hogg: CITE MODEL FITTING DOCUMENT},  

\section{Standard decision criteria}

If two models have similar freedom (suitably defined; though the
perceptive reader will notice that have avoided defining it\note{Hogg:
  What is the definition of model complexity?}), then the model with
the higher likelihood---or smaller value of $\chi^2$---is to be
preferred.  If the two models have different complexity or freedom,
then the model with the greater freedom will (in general) fit better;
in deciding between the models, the likelihood or $\chi^2$ should be
adjusted by something related to model complexity in order to penalize
the more free model for its freedom.

Despite all the warnings above about the ``number of parameters'', in
the case of linear models (predictions are linear combinations of the
parameters), and Gaussian noise of known variances, and no significant
prior information, the model complexity is just the number of
parameters $K$ (and in this same case, the logarithm of the likelihood
is just $-\chi^2/2$).  In this restrictive case---restrictive because
it is rare that models are perfectly linear, and when they are, it is
rare that the noise is Gaussian and well understood, and even when
that is the case, it is rare that the other implicit assumptions in
$H$ will be reasonable or justified---the penalty for model complexity
can be simply a constant times $K$ added to $\chi^2$.  The two
standard methods are AIC\note{Hogg cite AIC and spell it out and its
  assumptions}, where the comparison is made on the basis of $\chi^2 +
2\,K$, and BIC\note{Hogg cite BIC and spell it out and its
  asssumptions}, where it is $\chi^2 + K\,\ln N$ [CHECK], where $N$ is
the number of data points.  There is a lot to say about the
justifications for these methods---the AIC is based on predictive
power and the BIC is based on an approximation to the Bayesian
evidence (about which more below)---but we will spare the reader
because we \emph{do not recommend them}.

We don't recommend using the AIC, because in the linear case it
approximates cross-validation (HOGG DOES IT REALLY?), which---for a
linear problem---is fast to compute directly, with no approximation.
We don't recommend using the BIC, because in the linear case it
approximates the Bayesian evidence or marginalized likelihood (see
below).  Again, in the case of a linear problem, this is trivial and
fast to compute.  That is, both of these methods make approximations
that are only valid in cases in which the non-approximate calculation
is straightforward and inexpensive!  And when the models get
non-linear (so the number of parameters does \emph{not} precisely
describe the model freedom), and when the other assumptions (about
Gaussian errors and known variance and priors) become false, these
methods must behave unpredictably.\note{Well, these methods will
  always act ``predictably'' but only predictably in the sense that an
  expensive computational analysis will tell you important things
  about how they in fact behave, with a new analysis required for
  every new problem.}



...There is a deeper reason to dislike AIC and BIC and that is that
they make deep assumptions about your \emph{utility}... must be wrong,
because you don't even know what they are...

...In what circumstances \emph{does} it make sense to use them?  Fastness.

\section{Degrees of freedom}

One common---but seriously wrong---method for deciding between models
is to look at $\chi^2$ \emph{per degree of freedom} $\nu$, where the
very badly named\note{It should be called the ``number of degrees of
  constraint'' because it increases as the model becomes \emph{less}
  free.} ``number of degrees of freedom'' is defined to be the number
of data points $N$ minus the number of parameters $K$.  As with AIC
and BIC, if the model is a good fit to the data, and it is linear, and
the noise in the data points is Gaussian with known variances, then
$\chi^2/\nu$ ought to be near unity (within a few $1/\sqrt{\nu}$ of
unity).  Often this issue is raised when model selection is raised.
Unfortunately, this idea is \emph{not relevant} to model selection.
The comparison of $\chi^2$ to $\nu$ is appropriate in the context of
model \emph{checking}.  In model checking, the question is: ``Which
models do a good job of generating the data?''.  In model selection,
the question is ``Of this set of models, which should I choose?''.

It is confusing that two models might both be good at generating the
data---both generate a $\chi^2/\nu$ near enough unity to be
reasonable---but at the same time, one of them is enormously
preferred.  An example would be model A, with $\chi^2=45.3$, $\nu=44$,
and model B, with $\chi^2=51.1$, $\nu=43$, acting on the same data.
Both models have $\chi^2/\nu$ reasonable, but in fact by the standard
AIC or BIC criteria (which we have just disdained, but could
conceivably be applicable) model A would be greatly preferred, because
it has better $\chi^2$ \emph{and} fewer parameters (it had higher
$\nu$ so lower number of parameters).  The resolution of this
paradox---that both models can be ``fine'' but one is
``preferred''---is that the $\chi^2/\nu$ question is about
\emph{accuracy} and the model selection question is about
\emph{precision}.  The data might not \emph{rule out} a model but
still vastly prefer its competitor.

The much more common case is not this case.  The more common case is
that \emph{no} model is a good fit to the data---all models have bad
$\chi^2/\nu$---but we still need to make a call among them.  Once
again, the accuracy and precision questions are separate: Even if no
model does a good job of generating the data, you might still vastly
prefer one over the other.  And of course the sensible researcher
would not ignore the $\chi^2/\nu$ test: It indicates that the models
\emph{all} need improving.\note{...Stuff about all models being wrong,
  being approximate.  All models will fail the $\chi^2/\nu$ test if
  the data sets get large enough.  And so on...}

\section{Bayesian evidence or marginalized likelihood}

The only trivial---``trivial'' in the sense of introducing no new
ideas, not in the sense of ``easy''\note{Indeed, calculation of the
  Bayesian evidence or marginalized likelihood is usually extremely
  difficult and computationally expensive.}---Bayesian computation
relevant to model comparison is computation of the marginalized
likelihood or (badly named\note{The word ``evidence'' is uselessly
  overloaded.}) ``Bayesian evidence''.  This is the likelihood
integrated over all the parameter space, using the prior as the
integration measure (which is the real meaning of the prior
pdf\note{cite SOME OTHER DOCUMENT.}):
\begin{eqnarray}
p(\data\given H) &=& \int p(\data\given\pars, H)\,p(\pars\given H)\,\dd\pars
\quad,
\end{eqnarray}
where $p(\data\given H)$ is the marginalized likelihood, the
probability of the full data set $\data$ given the hypothesis or model
$H$, integrating over all the parameters $\pars$.

...Why it is \emph{so sensitive} to the priors?

...Why is it so expensive?  How do we compute it in practice?

Fundamentally, however---even if the prior sensitivity can be
swallowed and the computational expenses paid---the reason this long
and brutish calculation of marginalized likelihoods is not the right
way to \emph{decide among} models is that all it delivers is
marginalized likelihoods.  These are crucial ingredients in making
probabilistic statements about models, to be sure, but they fail to do
the important thing of \emph{deciding among} models.  The fundamental
reason is that raised at the beginning of this \documentname:
\emph{Decsions involve utility}.  Another way to put it is this:

It is true that marginalized likelihoods are the only things that a
true Bayesian can justifiably calculate, given multiple qualitatively
different models.  At the same time, a true Bayesian is a
probabilistic reasoner: He or she obtains \emph{probabilities} for
models, not \emph{choices} among models.  That is, these marginalized
likelihood values tell the true Bayesian not \emph{how to decide}
among the models but \emph{how to mix} the models!  That is, the best
model, at the end of a set of long, hard, prior-dependent marginalized
likelihood calculations is not the model with the largest marginalized
likelihood.  It is the model that is the mixture of all the models,
weighted by their overall model priors times their marginalized
likelihoods.  What does this mixture look like?  It treats the data as
being generated not by any one model, but rather from model $A$ with
some probability $P_A$, from model $B$ with some probability $P_B$,
and so on.

...get specific about calculation of these posterior mixing weights
and give an example with polynomials.

\section{Bayesian decision theory}

...It is all about utility.  And priors.

...What can frequentists do with utility?  They can only minimize
worst-case loss within some bag of models.  Or maximize best-case
gain (that isn't usually chosen, not sure why).

\section{Cross-validation}

The idea of cross-validation is that we will ask whether each model is
a good predictor of left-out data.  That is, whether the model---when
trained on all but a held-out portion of the data---does a good job of
predicting the held-out portion.  If the model doesn't have enough
freedom, it can't fit any data well, so it does badly by this measure.
When the model has too much freedom, it doesn't make very specific
predictions for new data (or it over-fits the data).  There is a
``sweet spot'' that balances quality of fit and model freedom, and
cross-validation finds it.

Let's imagine that the full data set $\data$ is made up of $N$ data
points or subsets of the data $D_n$.  Each of these $N$ subsets $D_n$
could comprise, say, ten data points or one data point.  All we need
is that there be $N$ of them, and that the $N$ be similar in ``size''
where that word will be defined below.  In symbols,
\begin{eqnarray}
\data &\equiv& \setofall{\data_n}{n=1}{N}
\quad.
\end{eqnarray}
From now on we will think of the data $\data_n$ as small
\emph{subsets} of the whole data set, not \emph{necessarily}
individual data points.  For each data subset $\data_n$ there is a
\emph{complementary} or \emph{leave-one-out} data subset
$\data_{[n]}$, which consists of all the data \emph{except} those in
$\data_n$.  In symbols,
\begin{eqnarray}
\data &=& \data_n \union \data_{[n]}
\\
\varnothing &=& \data_n \intersect \data_{[n]}
\quad,
\end{eqnarray}
where $\union$ is the set union operator, $\intersect$ is the set
intersection operator, and $\varnothing$ is the null set.

...continuing
\begin{eqnarray}
\pars_n &\leftarrow& \argmax_\pars p(\data_{[n]}\given\pars)
\\
\pcv(\data) &\leftarrow& \prod_{n=1}^N p(\data_n\given\pars_n)
\quad,
\end{eqnarray}

...or ``Bayesian'' version
\begin{eqnarray}
\pcv(\data) &\leftarrow& \prod_{n=1}^N \int p(\data_n\given\pars)\,p(\pars\given\data_{[n]})\,\dd\pars
\quad,
\end{eqnarray}
where inference proceeded as usual.  The sampling approximation works well here.

...definition of ``size'' for the $\data_n$.

...return to the utility point.  This method works well but makes
unspecified (and therefore wrong) assumptions about your utility.
That said, if you don't want to care about your utility, predictive
properties of the models sure are sensible surrogates.

\section{Informal methods} % WRONG NAME FOR THIS SECTION

...Informally, we often choose among models (effectively) by
permitting a freedom, fitting for a parameter that controls the
freedom, and finding that the parameter is very far from the null or
zero value.  This is not crazy!  It is related to AIC and BIC and etc.

...numbers of ``sigma''!  LHC results as an example.  Detecting a
faint star as an example.

...note the philosophical point (Gelman etc) that this might always be
the best thing to do: Find a parameter that joins the two models and
see where that parameter gets inferred.  Exoplanet example can return
here; parameter is the exoplanet mass.

\section{No need to decide?}

...Mix your models whenever you can; or carry them all forward.  Reminder
that the Bayesian evidence tells you how to \emph{mix} models, not
choose.

...Gelman point can return here too.

\clearpage
\markright{Notes}\theendnotes

\clearpage
\begin{thebibliography}{}\markright{References}
\bibitem[Bovy, Hogg, \& Roweis(2009)]{bovy}
  Bovy,~J., Hogg,~D.~W., \& Roweis, S.~T., 2009,
  Extreme deconvolution: inferring complete distribution functions from noisy, heterogeneous, and incomplete observations, 
  arXiv:0905.2979 [stat.ME]
\end{thebibliography}

\end{document}
